% Options for packages loaded elsewhere
\PassOptionsToPackage{unicode}{hyperref}
\PassOptionsToPackage{hyphens}{url}
\PassOptionsToPackage{dvipsnames,svgnames,x11names}{xcolor}
%
\documentclass[
  letterpaper,
  DIV=11,
  numbers=noendperiod]{scrreprt}

\usepackage{amsmath,amssymb}
\usepackage{lmodern}
\usepackage{iftex}
\ifPDFTeX
  \usepackage[T1]{fontenc}
  \usepackage[utf8]{inputenc}
  \usepackage{textcomp} % provide euro and other symbols
\else % if luatex or xetex
  \usepackage{unicode-math}
  \defaultfontfeatures{Scale=MatchLowercase}
  \defaultfontfeatures[\rmfamily]{Ligatures=TeX,Scale=1}
\fi
% Use upquote if available, for straight quotes in verbatim environments
\IfFileExists{upquote.sty}{\usepackage{upquote}}{}
\IfFileExists{microtype.sty}{% use microtype if available
  \usepackage[]{microtype}
  \UseMicrotypeSet[protrusion]{basicmath} % disable protrusion for tt fonts
}{}
\makeatletter
\@ifundefined{KOMAClassName}{% if non-KOMA class
  \IfFileExists{parskip.sty}{%
    \usepackage{parskip}
  }{% else
    \setlength{\parindent}{0pt}
    \setlength{\parskip}{6pt plus 2pt minus 1pt}}
}{% if KOMA class
  \KOMAoptions{parskip=half}}
\makeatother
\usepackage{xcolor}
\setlength{\emergencystretch}{3em} % prevent overfull lines
\setcounter{secnumdepth}{5}
% Make \paragraph and \subparagraph free-standing
\ifx\paragraph\undefined\else
  \let\oldparagraph\paragraph
  \renewcommand{\paragraph}[1]{\oldparagraph{#1}\mbox{}}
\fi
\ifx\subparagraph\undefined\else
  \let\oldsubparagraph\subparagraph
  \renewcommand{\subparagraph}[1]{\oldsubparagraph{#1}\mbox{}}
\fi

\usepackage{color}
\usepackage{fancyvrb}
\newcommand{\VerbBar}{|}
\newcommand{\VERB}{\Verb[commandchars=\\\{\}]}
\DefineVerbatimEnvironment{Highlighting}{Verbatim}{commandchars=\\\{\}}
% Add ',fontsize=\small' for more characters per line
\usepackage{framed}
\definecolor{shadecolor}{RGB}{241,243,245}
\newenvironment{Shaded}{\begin{snugshade}}{\end{snugshade}}
\newcommand{\AlertTok}[1]{\textcolor[rgb]{0.68,0.00,0.00}{#1}}
\newcommand{\AnnotationTok}[1]{\textcolor[rgb]{0.37,0.37,0.37}{#1}}
\newcommand{\AttributeTok}[1]{\textcolor[rgb]{0.40,0.45,0.13}{#1}}
\newcommand{\BaseNTok}[1]{\textcolor[rgb]{0.68,0.00,0.00}{#1}}
\newcommand{\BuiltInTok}[1]{\textcolor[rgb]{0.00,0.23,0.31}{#1}}
\newcommand{\CharTok}[1]{\textcolor[rgb]{0.13,0.47,0.30}{#1}}
\newcommand{\CommentTok}[1]{\textcolor[rgb]{0.37,0.37,0.37}{#1}}
\newcommand{\CommentVarTok}[1]{\textcolor[rgb]{0.37,0.37,0.37}{\textit{#1}}}
\newcommand{\ConstantTok}[1]{\textcolor[rgb]{0.56,0.35,0.01}{#1}}
\newcommand{\ControlFlowTok}[1]{\textcolor[rgb]{0.00,0.23,0.31}{#1}}
\newcommand{\DataTypeTok}[1]{\textcolor[rgb]{0.68,0.00,0.00}{#1}}
\newcommand{\DecValTok}[1]{\textcolor[rgb]{0.68,0.00,0.00}{#1}}
\newcommand{\DocumentationTok}[1]{\textcolor[rgb]{0.37,0.37,0.37}{\textit{#1}}}
\newcommand{\ErrorTok}[1]{\textcolor[rgb]{0.68,0.00,0.00}{#1}}
\newcommand{\ExtensionTok}[1]{\textcolor[rgb]{0.00,0.23,0.31}{#1}}
\newcommand{\FloatTok}[1]{\textcolor[rgb]{0.68,0.00,0.00}{#1}}
\newcommand{\FunctionTok}[1]{\textcolor[rgb]{0.28,0.35,0.67}{#1}}
\newcommand{\ImportTok}[1]{\textcolor[rgb]{0.00,0.46,0.62}{#1}}
\newcommand{\InformationTok}[1]{\textcolor[rgb]{0.37,0.37,0.37}{#1}}
\newcommand{\KeywordTok}[1]{\textcolor[rgb]{0.00,0.23,0.31}{#1}}
\newcommand{\NormalTok}[1]{\textcolor[rgb]{0.00,0.23,0.31}{#1}}
\newcommand{\OperatorTok}[1]{\textcolor[rgb]{0.37,0.37,0.37}{#1}}
\newcommand{\OtherTok}[1]{\textcolor[rgb]{0.00,0.23,0.31}{#1}}
\newcommand{\PreprocessorTok}[1]{\textcolor[rgb]{0.68,0.00,0.00}{#1}}
\newcommand{\RegionMarkerTok}[1]{\textcolor[rgb]{0.00,0.23,0.31}{#1}}
\newcommand{\SpecialCharTok}[1]{\textcolor[rgb]{0.37,0.37,0.37}{#1}}
\newcommand{\SpecialStringTok}[1]{\textcolor[rgb]{0.13,0.47,0.30}{#1}}
\newcommand{\StringTok}[1]{\textcolor[rgb]{0.13,0.47,0.30}{#1}}
\newcommand{\VariableTok}[1]{\textcolor[rgb]{0.07,0.07,0.07}{#1}}
\newcommand{\VerbatimStringTok}[1]{\textcolor[rgb]{0.13,0.47,0.30}{#1}}
\newcommand{\WarningTok}[1]{\textcolor[rgb]{0.37,0.37,0.37}{\textit{#1}}}

\providecommand{\tightlist}{%
  \setlength{\itemsep}{0pt}\setlength{\parskip}{0pt}}\usepackage{longtable,booktabs,array}
\usepackage{calc} % for calculating minipage widths
% Correct order of tables after \paragraph or \subparagraph
\usepackage{etoolbox}
\makeatletter
\patchcmd\longtable{\par}{\if@noskipsec\mbox{}\fi\par}{}{}
\makeatother
% Allow footnotes in longtable head/foot
\IfFileExists{footnotehyper.sty}{\usepackage{footnotehyper}}{\usepackage{footnote}}
\makesavenoteenv{longtable}
\usepackage{graphicx}
\makeatletter
\def\maxwidth{\ifdim\Gin@nat@width>\linewidth\linewidth\else\Gin@nat@width\fi}
\def\maxheight{\ifdim\Gin@nat@height>\textheight\textheight\else\Gin@nat@height\fi}
\makeatother
% Scale images if necessary, so that they will not overflow the page
% margins by default, and it is still possible to overwrite the defaults
% using explicit options in \includegraphics[width, height, ...]{}
\setkeys{Gin}{width=\maxwidth,height=\maxheight,keepaspectratio}
% Set default figure placement to htbp
\makeatletter
\def\fps@figure{htbp}
\makeatother
\newlength{\cslhangindent}
\setlength{\cslhangindent}{1.5em}
\newlength{\csllabelwidth}
\setlength{\csllabelwidth}{3em}
\newlength{\cslentryspacingunit} % times entry-spacing
\setlength{\cslentryspacingunit}{\parskip}
\newenvironment{CSLReferences}[2] % #1 hanging-ident, #2 entry spacing
 {% don't indent paragraphs
  \setlength{\parindent}{0pt}
  % turn on hanging indent if param 1 is 1
  \ifodd #1
  \let\oldpar\par
  \def\par{\hangindent=\cslhangindent\oldpar}
  \fi
  % set entry spacing
  \setlength{\parskip}{#2\cslentryspacingunit}
 }%
 {}
\usepackage{calc}
\newcommand{\CSLBlock}[1]{#1\hfill\break}
\newcommand{\CSLLeftMargin}[1]{\parbox[t]{\csllabelwidth}{#1}}
\newcommand{\CSLRightInline}[1]{\parbox[t]{\linewidth - \csllabelwidth}{#1}\break}
\newcommand{\CSLIndent}[1]{\hspace{\cslhangindent}#1}

\KOMAoption{captions}{tableheading}
\makeatletter
\makeatother
\makeatletter
\@ifpackageloaded{bookmark}{}{\usepackage{bookmark}}
\makeatother
\makeatletter
\@ifpackageloaded{caption}{}{\usepackage{caption}}
\AtBeginDocument{%
\ifdefined\contentsname
  \renewcommand*\contentsname{Table des matières}
\else
  \newcommand\contentsname{Table des matières}
\fi
\ifdefined\listfigurename
  \renewcommand*\listfigurename{Liste des Figures}
\else
  \newcommand\listfigurename{Liste des Figures}
\fi
\ifdefined\listtablename
  \renewcommand*\listtablename{Liste des Tables}
\else
  \newcommand\listtablename{Liste des Tables}
\fi
\ifdefined\figurename
  \renewcommand*\figurename{Figure}
\else
  \newcommand\figurename{Figure}
\fi
\ifdefined\tablename
  \renewcommand*\tablename{Table}
\else
  \newcommand\tablename{Table}
\fi
}
\@ifpackageloaded{float}{}{\usepackage{float}}
\floatstyle{ruled}
\@ifundefined{c@chapter}{\newfloat{codelisting}{h}{lop}}{\newfloat{codelisting}{h}{lop}[chapter]}
\floatname{codelisting}{Listing}
\newcommand*\listoflistings{\listof{codelisting}{Liste des Listings}}
\makeatother
\makeatletter
\@ifpackageloaded{caption}{}{\usepackage{caption}}
\@ifpackageloaded{subcaption}{}{\usepackage{subcaption}}
\makeatother
\makeatletter
\@ifpackageloaded{tcolorbox}{}{\usepackage[many]{tcolorbox}}
\makeatother
\makeatletter
\@ifundefined{shadecolor}{\definecolor{shadecolor}{rgb}{.97, .97, .97}}
\makeatother
\makeatletter
\makeatother
\ifLuaTeX
\usepackage[bidi=basic]{babel}
\else
\usepackage[bidi=default]{babel}
\fi
\babelprovide[main,import]{french}
% get rid of language-specific shorthands (see #6817):
\let\LanguageShortHands\languageshorthands
\def\languageshorthands#1{}
\ifLuaTeX
  \usepackage{selnolig}  % disable illegal ligatures
\fi
\IfFileExists{bookmark.sty}{\usepackage{bookmark}}{\usepackage{hyperref}}
\IfFileExists{xurl.sty}{\usepackage{xurl}}{} % add URL line breaks if available
\urlstyle{same} % disable monospaced font for URLs
\hypersetup{
  pdftitle={Principes, procédures et tips pour traiter les données dans R},
  pdfauthor={PHilippe Gay; Nicolas Bressoud},
  pdflang={fr},
  colorlinks=true,
  linkcolor={blue},
  filecolor={Maroon},
  citecolor={Blue},
  urlcolor={Blue},
  pdfcreator={LaTeX via pandoc}}

\title{Principes, procédures et tips pour traiter les données dans R}
\author{PHilippe Gay \and Nicolas Bressoud}
\date{19/11/2022}

\begin{document}
\maketitle
\ifdefined\Shaded\renewenvironment{Shaded}{\begin{tcolorbox}[borderline west={3pt}{0pt}{shadecolor}, enhanced, breakable, sharp corners, frame hidden, boxrule=0pt, interior hidden]}{\end{tcolorbox}}\fi

\renewcommand*\contentsname{Table des matières}
{
\hypersetup{linkcolor=}
\setcounter{tocdepth}{2}
\tableofcontents
}
\bookmarksetup{startatroot}

\hypertarget{pruxe9face}{%
\chapter*{Préface}\label{pruxe9face}}
\addcontentsline{toc}{chapter}{Préface}

\markboth{Préface}{Préface}

Pourquoi ce livre ? Contexte.

\bookmarksetup{startatroot}

\hypertarget{comment-lire-ce-livre}{%
\chapter*{Comment lire ce livre}\label{comment-lire-ce-livre}}
\addcontentsline{toc}{chapter}{Comment lire ce livre}

\markboth{Comment lire ce livre}{Comment lire ce livre}

philosophie, utilisation

\bookmarksetup{startatroot}

\hypertarget{environnement-de-travail}{%
\chapter{Environnement de travail}\label{environnement-de-travail}}

Notre configuration de travail implique:

\begin{itemize}
\tightlist
\item
  \emph{GitHub} qui agit comme lieu principal de \textbf{stockage}, de
  \textbf{communication} de codes ou/et résultats, de
  \textbf{versioning}.
\item
  \emph{RStudio} dont le fonctionnement avec \texttt{bookdown} est
  particulièrement appréciable
\item
  \emph{Sublime Text} qui doit être mieux connue mais permet de
  travailler sur des bouts de code avec une syntaxe couleurs et facilite
  ainsi le \emph{copier/coller} d'anciens projets vers de nouveaux.
\end{itemize}

\hypertarget{github-et-structure-des-dossiers}{%
\section{GitHub et structure des
dossiers}\label{github-et-structure-des-dossiers}}

On dispose d'un compte chez \emph{GitHub}. Le compte est lié à 3
machines (2 macs et un pc). Le \emph{commit} et le \emph{push} se
réalisent depuis RStudio directement. Sur chaque machine, on s'offre
tout de même une solution \emph{user friendly} avec \emph{GitHub
Desktop}.

Sur chaque machine, on dispose d'une structure des dossiers de type
\emph{Documents \textgreater{} GitHub \textgreater{} r-projects}
(attention à la casse).

Le dossier \emph{r-projects} contient autant de sous-dossiers que de
projets à analyser.

Chaque sous-dossier a un nom structuré ainsi (attention à la casse) :
\emph{contexteannée\_initialesauteurprincipal} (par exemple :
\emph{dupp2019\_cr} ou \emph{crips2019\_lv}).

Dans chaque sous-dossier:

\begin{itemize}
\tightlist
\item
  un fichier \emph{Rproj} est disponible et il porte le nom du
  sous-dossier
\item
  un fichier présentant les données brutes (raw) est disponible (en
  lecture seule de préférence) et il porte un titre de la forme
  \emph{nomsousdossier\_raw}. Il peut être en format \emph{.csv} ou
  \emph{.xlsx}. Si les données brutes sont sur plusieurs fichiers (dans
  le cas de plusieurs temps de mesure, par exemple, on les distingue
  avec l'ajout *\_t1*, \ldots)
\item
  un fichier de script \emph{.R} intitulé \emph{nomsousdossier\_script}
\item
  un fichier Rmarkdown \emph{.Rmd} intitulé
  \emph{nomsousdossier\_rapport} et rédigé en parallèle du script. Ce
  rapport général appelle le script pour réaliser les sorties.
\end{itemize}

\begin{quote}
On ne s'est pas encore déterminé sur les bonnes pratiques pour la
constitution du rapport en RMarkdown : Est-ce OK et satisfaisant de
``simplement'' rappeler le script et uniquement ``afficher'' les objets
?
\end{quote}

\begin{itemize}
\tightlist
\item
  les sorties de type \emph{HTML}, \emph{PDF}, ou image (\emph{png},
  \emph{svg}, \ldots) n'obéissent pas à des règles précises.
\item
  plusieurs rapports peuvent coexister en fonction des destinataires ;
  ils sont tous une adaptation du rapport ``master'' décrit plus haut.
\end{itemize}

L'intérêt est d'avoir à tout moment sur GitHub une vision claire des
modifications réalisées à travers les différentes étapes de mise à jour
des fichier (traçabilité de la démarche). De plus, une attention
particulière est accordée à l'écriture d'un code avec une grammaire (le
plus possible) conventionnelle qui est lisible et commenté, que ce soit
dans le script ou dans le rapport en RMarkdown.

Le script et le rapport se rédigent en parallèle.

\hypertarget{ruxe9daction-du-code-et-grammaire}{%
\section{Rédaction du code et
grammaire}\label{ruxe9daction-du-code-et-grammaire}}

Le code suivant est, à notre sens, un exemple de bonne pratique car :

\begin{itemize}
\tightlist
\item
  des titres mis en évidence structurent le code
\item
  les commentaires sont présents; ils se veulent précis et concis
\item
  des espaces facilitent la lecture
\item
  le code n'est pas \emph{- à notre connaissance -} inutilement
  répétitif
\end{itemize}

\begin{Shaded}
\begin{Highlighting}[]
\DocumentationTok{\#\#\#\#\#\#\#\#\#\#\#\#\#\#\#}
\CommentTok{\#visualisation\#}
\DocumentationTok{\#\#\#\#\#\#\#\#\#\#\#\#\#\#\#}

\CommentTok{\#score échelle HBSC}

\NormalTok{vis\_hbs }\OtherTok{\textless{}{-}}\NormalTok{ d\_long\_paired }\SpecialCharTok{\%\textgreater{}\%} 
  \FunctionTok{ggplot}\NormalTok{() }\SpecialCharTok{+}
  \FunctionTok{aes}\NormalTok{(}\AttributeTok{x =}\NormalTok{ temps, }\AttributeTok{color =}\NormalTok{ group, }\AttributeTok{y =}\NormalTok{ hbs\_sco) }\SpecialCharTok{+}
  \FunctionTok{geom\_boxplot}\NormalTok{(}\AttributeTok{alpha =}\NormalTok{ .}\DecValTok{5}\NormalTok{) }\SpecialCharTok{+}
  \FunctionTok{geom\_jitter}\NormalTok{(}\AttributeTok{size =} \DecValTok{5}\NormalTok{, }\AttributeTok{alpha =}\NormalTok{ .}\DecValTok{5}\NormalTok{, }\AttributeTok{position =} \FunctionTok{position\_jitterdodge}\NormalTok{(}\AttributeTok{dodge.width=}\NormalTok{.}\DecValTok{7}\NormalTok{, }\AttributeTok{jitter.width =}\NormalTok{ .}\DecValTok{2}\NormalTok{)) }\SpecialCharTok{+}
  \FunctionTok{stat\_summary}\NormalTok{(}\AttributeTok{fun =}\NormalTok{ mean, }\AttributeTok{geom =} \StringTok{"point"}\NormalTok{, }\AttributeTok{size =} \DecValTok{3}\NormalTok{, }\AttributeTok{shape =} \DecValTok{4}\NormalTok{) }\SpecialCharTok{+}
  \FunctionTok{stat\_summary}\NormalTok{(}\AttributeTok{fun =}\NormalTok{ mean, }\FunctionTok{aes}\NormalTok{(}\AttributeTok{group =}\NormalTok{ group), }\AttributeTok{geom =} \StringTok{"line"}\NormalTok{) }\SpecialCharTok{+}
  \FunctionTok{labs}\NormalTok{(}\AttributeTok{title =} \StringTok{"Mesure des CPS"}\NormalTok{, }\AttributeTok{y =} \StringTok{"Score au HBSC"}\NormalTok{) }\SpecialCharTok{+}
  \FunctionTok{theme}\NormalTok{(}\AttributeTok{plot.title =} \FunctionTok{element\_text}\NormalTok{(}\AttributeTok{hjust =} \FloatTok{0.5}\NormalTok{)) }\SpecialCharTok{+}
  \FunctionTok{scale\_color\_brewer}\NormalTok{(}\StringTok{"Groupe"}\NormalTok{, }\AttributeTok{palette =} \StringTok{"Set1"}\NormalTok{)}
\end{Highlighting}
\end{Shaded}

Un autre exemple illustre cette grammaire. On rajoute quelques espaces
pour faciliter la lecture et repérer les structures \emph{- parfois
nécessairement -} répétitives :

\begin{quote}
On aimerait toutefois savoir comment éviter ces répétitions. Notamment
quand on génère 10 graphiques qui ne varient, dans le code, qu'au niveau
de l'axe Y et le titre, par exemple.
\end{quote}

\begin{Shaded}
\begin{Highlighting}[]
\CommentTok{\#Création des moyennes de chaque questionnaire pour chaque observation}

\NormalTok{d\_long }\OtherTok{\textless{}{-}}\NormalTok{ d\_long }\SpecialCharTok{\%\textgreater{}\%} 
  \FunctionTok{mutate}\NormalTok{(}\AttributeTok{hbs\_sco =} \FunctionTok{rowMeans}\NormalTok{(}\FunctionTok{select}\NormalTok{(.,}\FunctionTok{starts\_with}\NormalTok{(}\StringTok{"hbs"}\NormalTok{)),}\AttributeTok{na.rm =}\NormalTok{T),}
         \AttributeTok{pec\_sco =} \FunctionTok{rowMeans}\NormalTok{(}\FunctionTok{select}\NormalTok{(.,}\FunctionTok{starts\_with}\NormalTok{(}\StringTok{"pec"}\NormalTok{)),}\AttributeTok{na.rm =}\NormalTok{T),}
         \AttributeTok{be\_sco  =} \FunctionTok{rowMeans}\NormalTok{(}\FunctionTok{select}\NormalTok{(.,}\FunctionTok{starts\_with}\NormalTok{(}\StringTok{"be"}\NormalTok{)) ,}\AttributeTok{na.rm =}\NormalTok{T),}
         \AttributeTok{est\_sco =} \FunctionTok{rowMeans}\NormalTok{(}\FunctionTok{select}\NormalTok{(.,}\FunctionTok{starts\_with}\NormalTok{(}\StringTok{"est"}\NormalTok{)),}\AttributeTok{na.rm =}\NormalTok{T),}
         \AttributeTok{cli\_sco =} \FunctionTok{rowMeans}\NormalTok{(}\FunctionTok{select}\NormalTok{(.,}\FunctionTok{starts\_with}\NormalTok{(}\StringTok{"cli"}\NormalTok{)),}\AttributeTok{na.rm =}\NormalTok{T),}
         \AttributeTok{sou\_sco =} \FunctionTok{rowMeans}\NormalTok{(}\FunctionTok{select}\NormalTok{(.,}\FunctionTok{starts\_with}\NormalTok{(}\StringTok{"sou"}\NormalTok{)),}\AttributeTok{na.rm =}\NormalTok{T),}
         \AttributeTok{mot\_sco =} \FunctionTok{rowMeans}\NormalTok{(}\FunctionTok{select}\NormalTok{(.,}\FunctionTok{starts\_with}\NormalTok{(}\StringTok{"hbs"}\NormalTok{)),}\AttributeTok{na.rm =}\NormalTok{T))}
\end{Highlighting}
\end{Shaded}

\textbf{Une source pour la grammaire du codage peut être consultée à
l'adresse :}
https://www.inwt-statistics.com/read-blog/inwts-guidelines-for-r-code.html

\hypertarget{rstudio-et-packages}{%
\section{RStudio et packages}\label{rstudio-et-packages}}

On dispose en l'état de la version 4.0.2 de R ainsi que de la version
1.3.959 de RStudio.

Sur les macs, différents ajouts comme Xquartz ou des modules liés à
LaTeX sont également installés.

\begin{quote}
Mais honnêtement, on a perdu de vue leur rôle plus ou moins nécessaire
sachant que LaTeX est certes nécessaire pour les sorties PDF sans que
toutes les extensions liées à LaTex le soient\ldots{} Ce sera à
clarifier au prochain clean install.
\end{quote}

Pour la version PC, on a installé tinyTEX `tinytex::install\_tinytex()'
directement `TinyTeX to C:\Users\nbr\AppData\Roaming/TinyTeX' depuis la
console. On rappelle que \texttt{LaTeX} est nécessaire à la génération
de \emph{PDF}.

Dans R, les packages suivants sont installés:

\begin{itemize}
\tightlist
\item
  \texttt{tidyverse}, suite de packages pour travailler de manière
  \emph{tidy} (bien rangé) : \texttt{ggplot2}, \texttt{dplyr},
  \texttt{tidyr}, \texttt{readr}, \texttt{purrr}, \texttt{tibble},
  \texttt{stringr}, \texttt{forcats}.
\item
  \texttt{readxl}, permet de lire et importer les fichiers .xlsx.
\item
  \texttt{bookdown}, permet de réaliser à peu de frais le présent livre
\end{itemize}

En principe, pour des raisons d'élégance du code, on cherche à limiter
la sur-installation de nouveaux packages. Il s'agit d'explorer ce que
les packages installés ont à offrir avant de courir sur d'autres
fonctions vues sur le web.

\hypertarget{sublime-text-et-packages}{%
\section{Sublime Text et Packages}\label{sublime-text-et-packages}}

L'utilisation de Sublime Tex est ergonomique. Les packages du logiciel
permettent de travailler comme R Studio, avec toutefois un environnement
plus aéré et propice à la rédaction avec des codes couleurs agréables.
Il est complémentaire à RStudio mais peut carrément le remplacer pour
certaines courtes étapes de rédaction.

Les packages installés sont:

\begin{itemize}
\tightlist
\item
  sur PC : R-Box, R-IDE, LSP
\item
  sur MAC : R-Box, SendCode
\end{itemize}

\begin{quote}
On doit encore clarifier comment lier sur PC et MAC GitHub. Mais ce
n'est pas prioritaire. GitHub sur Sublime Text ? Emmet, git, sublime
github. https://gist.github.com/KedrikG/f7b955dc371b1204ec76ce862e2dcd2e
\end{quote}

\hypertarget{scrivener-3-zotero-betterbibtext---ruxe9daction-darticles-longs}{%
\section{Scrivener 3 + Zotero + BetterBibText - Rédaction d'articles
longs}\label{scrivener-3-zotero-betterbibtext---ruxe9daction-darticles-longs}}

On doit déterminer comment travailler en \texttt{RMarkdown} pour des
articles longs via \texttt{Scrivener\ 3}. Ce logiciel a l'avantage de
facilement découper l'article en plusieurs zones de travail et
\emph{fusionner} le tout à l'export. De plus, on peut lier dynamiquement
la rédaction à Zotero via un fichier \texttt{Bibtex} ce qui est très
intéressant.

On pense aussi à rassembler les éléments rédigés dans Scrivener et
compiler le tout avec \texttt{Bookdown} sous R.

Un hypothétique *workflow'' serait : script dans R \textgreater{}
ébauche de rapport dans R (RMarkdown) \textgreater{} copier/coller de
l'ébauche de rapport dans Scrivener \textgreater{} rédaction dans
Scrivener en compatibilité avec \texttt{Bookdown} \textgreater{} export
de Scrivener dans R \textgreater{} préparation de la sortie avec
\texttt{Bookdown}.

Mais ce n'est pas optimal. Notre souhait est de pouvoir, p.ex., modifier
une donnée dans la source des données (le fichier brut) et cliquer sur
un (voire deux) boutons pour mettre à jour l'article final ! On n'y est
pas encore !!

L'organisation de la rédaction est un gros chantier qui n'est pas du
tout structuré en l'état.

\bookmarksetup{startatroot}

\hypertarget{pruxe9paration-des-donnuxe9es}{%
\chapter{Préparation des données}\label{pruxe9paration-des-donnuxe9es}}

Au sein de ce chapitre, on cherche à atteindre le double objectif de/d'
:

\begin{itemize}
\tightlist
\item
  expliciter une manière que l'on espère \emph{élégante} et
  \emph{efficace} \emph{- Le minimum d'effort pour le maximum
  d'efficacité) -} de préparer les données
\item
  se conformer aux apprentissages et pratiques en traitement des données
  (et progresser)
\end{itemize}

On cherche dans la mesure du possible à toujours travailler avec des
\emph{tidy datas} et à utiliser en priorité les fonctions de la série de
packages de \texttt{tidyverse}.

\hypertarget{bonnes-pratiques-dans-le-systuxe8me-de-ruxe9colte-des-donnuxe9es-qualtrics}{%
\section{Bonnes pratiques dans le système de récolte des données
(Qualtrics,
\ldots)}\label{bonnes-pratiques-dans-le-systuxe8me-de-ruxe9colte-des-donnuxe9es-qualtrics}}

Chaque participant·e reçoit un \emph{id} partiellement anonymisé. Il est
fourni par l'équipe de recherche (ce n'est pas construit par le ou la
participant·e). L'\emph{id} ne contient que des chiffres pour simplifier
le problème des majuscules/minuscules à la saisie et éviter les
confusions entre le \emph{zéro} et la lettre \emph{o} (oui oui, c'est du
vécu). Ceci signifie que l'\emph{id} dispose d'une structure qui permet
de/d' :

\begin{itemize}
\tightlist
\item
  évaluer son authenticité
\item
  faciliter la catégorisation des données
\item
  repérer efficacement les \emph{id} et leur catégorisation dans les
  traitement ultérieurs
\end{itemize}

\begin{quote}
Exemple : un \emph{id} comme \emph{12.47.694} est structuré comme suit
\emph{groupe12.constantearbitraire47.nombrealéatoireà3chiffres.}
\end{quote}

Chaque \emph{id} est \textbf{vérifié} dans Qualtrics, ce qui signifie
qu'un·e participant·e doit confirmer l'\emph{id} pour que le système
passe à la suite. On relève aussi l'intérêt de demander à
l'utilisateur·trice de \textbf{confirmer son groupe d'appartenance} en
cochant un facteur dans une \textbf{liste imposée} dans Qualtrics. Ceci
a l'avantage de repérer les saisies fantasques (une personne met un
\emph{id} bidon / un·e participant·e s'inscrit avec le bon \emph{id}
mais dans le mauvais groupe, \ldots). Dans la préparation des données,
il semble que l'on gagne ainsi un temps important pour trier les données
valables. La confiance dans les données s'en trouve augmentée. On relève
toutefois une faiblesse : le code n'est pas 100\% anonyme et on peut
toujours remonter à un groupe de participant·es. Il faut alors en amont
s'engager à détruire le document qui a permis la génération des
\emph{id}.

Chaque questionnaire est identifié par 3 premières lettre
significatives, le nombre d'items et un \texttt{\_}:

\begin{quote}
Exemple : Questionnaire sur la motivation scolaire en 13 items est
identifié par \texttt{mot13}.
\end{quote}

Ensuite, chacun des items est proprement nommé par ordre d'apparition,
ce qui est fondamental, notamment pour le traitement des \textbf{items
inversés} :

\begin{quote}
Exemple : \texttt{mot13\_1}, \texttt{mot13\_2}, \ldots,
\texttt{mot13\_13} (il semble que Qualtrics ajoute par défaut le
\emph{underscore\_} )
\end{quote}

\hypertarget{premiuxe8res-uxe9tapes-dans-r-en-principe-les-noms-des-variables-sont-sains}{%
\section{Premières étapes dans R (en principe, les noms des variables
sont
sains)}\label{premiuxe8res-uxe9tapes-dans-r-en-principe-les-noms-des-variables-sont-sains}}

On commence par le chargement des packages et l'importation du ou des
fichiers de données. Dans cet exemple, on prend le cas (plus complexe)
où nous devons gérer et associer deux \texttt{data\ frames}.

\begin{Shaded}
\begin{Highlighting}[]
\FunctionTok{library}\NormalTok{(tidyverse)}
\FunctionTok{library}\NormalTok{(readxl)}

\CommentTok{\# Importation des données Qualtrics à disposition {-}{-}{-}{-}}
\NormalTok{d\_t1\_raw }\OtherTok{\textless{}{-}} \FunctionTok{read\_excel}\NormalTok{(}\StringTok{"crips2019\_lv\_raw\_t1.xlsx"}\NormalTok{)}
\NormalTok{d\_t2\_raw }\OtherTok{\textless{}{-}} \FunctionTok{read\_excel}\NormalTok{(}\StringTok{"crips2019\_lv\_raw\_t2.xlsx"}\NormalTok{)}
\end{Highlighting}
\end{Shaded}

\hypertarget{ajout-de-la-variable-de-temps-dans-chaque-df-cas-simple}{%
\subsection{ajout de la variable de temps dans chaque df (cas
simple)}\label{ajout-de-la-variable-de-temps-dans-chaque-df-cas-simple}}

Chaque fichier représente un temps de mesure. Il nous suffit d'ajouter
une variable indiquant cette information.

Dans cet exemple, on règle en même temps le problème des \emph{id} qui
contiennent éventuellement des majuscules.

\begin{Shaded}
\begin{Highlighting}[]
\CommentTok{\#Ajout de la variable temps sur chaque df et normalisation des id en minuscule.}

\NormalTok{d\_t1 }\OtherTok{\textless{}{-}}\NormalTok{ d\_t1\_raw }\SpecialCharTok{\%\textgreater{}\%} 
  \FunctionTok{mutate}\NormalTok{(}\AttributeTok{temps=}\StringTok{"temps 1"}\NormalTok{,}
         \AttributeTok{id=}\FunctionTok{tolower}\NormalTok{(id)) }

\NormalTok{d\_t2 }\OtherTok{\textless{}{-}}\NormalTok{ d\_t2\_raw }\SpecialCharTok{\%\textgreater{}\%} 
  \FunctionTok{mutate}\NormalTok{(}\AttributeTok{temps=}\StringTok{"temps 2"}\NormalTok{,}
         \AttributeTok{id=}\FunctionTok{tolower}\NormalTok{(id))}
\end{Highlighting}
\end{Shaded}

\hypertarget{catuxe9gorisation-de-la-variable-de-temps-cas-avec-horodatage-pruxe9alable}{%
\subsection{catégorisation de la variable de temps (cas avec horodatage
préalable)}\label{catuxe9gorisation-de-la-variable-de-temps-cas-avec-horodatage-pruxe9alable}}

Quand on possède un unique fichier regroupant toutes les observations,
on peut catégoriser les données à partir de l'horodateur.

\begin{Shaded}
\begin{Highlighting}[]
\CommentTok{\#modification de la variable "RecordedDate" en "date" en temps 1 et temps 2 tout en filtrant les id enregistrés hors temps 1 et temps 2.}
\CommentTok{\#mais on commence par la renommer.}

\NormalTok{d }\OtherTok{\textless{}{-}}\NormalTok{ d }\SpecialCharTok{\%\textgreater{}\%} 
  \FunctionTok{rename}\NormalTok{(}\AttributeTok{date =}\NormalTok{ RecordedDate) }\SpecialCharTok{\%\textgreater{}\%}  \CommentTok{\#dans cet ordre.}
  \FunctionTok{mutate}\NormalTok{(}\AttributeTok{id =} \FunctionTok{tolower}\NormalTok{(id)) }\CommentTok{\#gestion de la casse.}

\NormalTok{d }\OtherTok{\textless{}{-}}\NormalTok{ d }\SpecialCharTok{\%\textgreater{}\%} 
  \FunctionTok{mutate}\NormalTok{(}
    \AttributeTok{date =} \FunctionTok{case\_when}\NormalTok{(date }\SpecialCharTok{\textgreater{}=} \FunctionTok{as.POSIXct}\NormalTok{(}\StringTok{"01.08.2019"}\NormalTok{, }\AttributeTok{format=}\StringTok{"\%d.\%m.\%Y"}\NormalTok{, }\AttributeTok{tz=}\StringTok{"utc"}\NormalTok{) }\SpecialCharTok{\&}\NormalTok{ date }\SpecialCharTok{\textless{}=} \FunctionTok{as.POSIXct}\NormalTok{(}\StringTok{"31.08.2019"}\NormalTok{, }\AttributeTok{format=}\StringTok{"\%d.\%m.\%Y"}\NormalTok{, }\AttributeTok{tz=}\StringTok{"utc"}\NormalTok{) }\SpecialCharTok{\textasciitilde{}} \StringTok{"temps 1"}\NormalTok{,}
\NormalTok{                     date }\SpecialCharTok{\textgreater{}=} \FunctionTok{as.POSIXct}\NormalTok{(}\StringTok{"01.06.2020"}\NormalTok{, }\AttributeTok{format=}\StringTok{"\%d.\%m.\%Y"}\NormalTok{, }\AttributeTok{tz=}\StringTok{"utc"}\NormalTok{) }\SpecialCharTok{\&}\NormalTok{ date }\SpecialCharTok{\textless{}=} \FunctionTok{as.POSIXct}\NormalTok{(}\StringTok{"30.06.2020"}\NormalTok{, }\AttributeTok{format=}\StringTok{"\%d.\%m.\%Y"}\NormalTok{, }\AttributeTok{tz=}\StringTok{"utc"}\NormalTok{) }\SpecialCharTok{\textasciitilde{}} \StringTok{"temps 2"}\NormalTok{,}
                     \ConstantTok{TRUE} \SpecialCharTok{\textasciitilde{}} \StringTok{"autre temps"}\NormalTok{)) }\CommentTok{\#on privilégie case\_when car on a 3 conditions et on va gérer les dates.}

\CommentTok{\#Au passage, R a modifié le type de variable "date". On le laisser respirer... et on filtre... (si j\textquotesingle{}intègre filter dans le pipe, ça bug...)}

\NormalTok{d }\OtherTok{\textless{}{-}}\NormalTok{ d }\SpecialCharTok{\%\textgreater{}\%} \FunctionTok{filter}\NormalTok{(date }\SpecialCharTok{==}\StringTok{"temps 1"} \SpecialCharTok{|}\NormalTok{ date }\SpecialCharTok{==} \StringTok{"temps 2"}\NormalTok{)}
\end{Highlighting}
\end{Shaded}

\begin{quote}
On est pas encore confiant sur la qualité du code ci-dessus.
\end{quote}

\hypertarget{cruxe9ation-dun-data-frame-unique}{%
\subsection{création d'un data frame
unique}\label{cruxe9ation-dun-data-frame-unique}}

A ce stade, à partir du cas simple, il ne semble pas y avoir de raison
au maintien de 2 df différents. On peut créer un df unique à l'aide,
simplement, de \texttt{bind\_rows}. Cela nous évite de doubler les
manipulations au temps 1 et au temps 2.

Au préalable, on s'est assuré que les noms des variables correspondaient
à notre nomenclature, et que l'ordre des questions dans chaque
questionnaire était le même au \emph{temps 1} et au \emph{temps 2}.

Les éventuelles nouvelles variables du temps 2 sont bien traitées grâce
à l'ajout de \texttt{NA} pour les observations du temps 1.

\begin{Shaded}
\begin{Highlighting}[]
\CommentTok{\#Mise en formation long sur un df}

\NormalTok{d\_long }\OtherTok{\textless{}{-}} \FunctionTok{bind\_rows}\NormalTok{(d\_t1,d\_t2)}
\end{Highlighting}
\end{Shaded}

\begin{quote}
On choisit la mise en format \emph{long} (au lieu de \emph{wide}) pour
correspondre au attentes du traitement \emph{tidy} des données. Cela
signifie que chaque ligne est une observation et chaque colonne est une
variable. Ceci implique que chaque participant·e se retrouve dans deux
lignes : une concernant la modalité (facteur) \textbf{temps 1} et
l'autre selon la modalité \textbf{temps 2}. C'était déroutant au début
mais on a adopté cette manière de faire.
\end{quote}

\hypertarget{gestion-des-items-uxe0-inverser}{%
\subsection{gestion des items à
inverser}\label{gestion-des-items-uxe0-inverser}}

Les items à inverser sont traités en étant strictement remplacés via la
fonction \texttt{mutate()} dans le cadre de l'enregistrement d'un
nouveau \texttt{data\ frame}. R permet de remonter les changements donc
cet écrasement n'empêche pas la vérification des processus, étape par
étape.

\begin{Shaded}
\begin{Highlighting}[]
\CommentTok{\#Recondage des variables au score inversé}

\NormalTok{d\_long }\OtherTok{\textless{}{-}}\NormalTok{ d\_long }\SpecialCharTok{\%\textgreater{}\%} 
  \FunctionTok{mutate}\NormalTok{(}\AttributeTok{hbs20\_\_7 =} \DecValTok{6} \SpecialCharTok{{-}}\NormalTok{ hbs20\_\_7,}
         \AttributeTok{pec5\_\_5 =} \DecValTok{6} \SpecialCharTok{{-}}\NormalTok{ pec5\_\_5,}
         \AttributeTok{be8\_\_4 =} \DecValTok{8} \SpecialCharTok{{-}}\NormalTok{ be8\_\_4,}
         \AttributeTok{be8\_\_5 =} \DecValTok{8} \SpecialCharTok{{-}}\NormalTok{ be8\_\_5,}
         \AttributeTok{be8\_\_8 =} \DecValTok{8} \SpecialCharTok{{-}}\NormalTok{ be8\_\_8,}
         \AttributeTok{est10\_\_3 =} \DecValTok{5} \SpecialCharTok{{-}}\NormalTok{ est10\_\_3,}
         \AttributeTok{est10\_\_5 =} \DecValTok{5} \SpecialCharTok{{-}}\NormalTok{ est10\_\_5,}
         \AttributeTok{est10\_\_7 =} \DecValTok{5} \SpecialCharTok{{-}}\NormalTok{ est10\_\_7,}
         \AttributeTok{est10\_\_10 =} \DecValTok{5} \SpecialCharTok{{-}}\NormalTok{ est10\_\_10,}
         \AttributeTok{sou13\_\_6 =} \DecValTok{6} \SpecialCharTok{{-}}\NormalTok{ sou13\_\_6,}
         \AttributeTok{sou13\_\_9 =} \DecValTok{6} \SpecialCharTok{{-}}\NormalTok{ sou13\_\_9,}
         \AttributeTok{mot16\_\_4 =} \DecValTok{8} \SpecialCharTok{{-}}\NormalTok{ mot16\_\_4,}
         \AttributeTok{mot16\_\_8 =} \DecValTok{8} \SpecialCharTok{{-}}\NormalTok{ mot16\_\_8,}
         \AttributeTok{mot16\_\_12 =} \DecValTok{8} \SpecialCharTok{{-}}\NormalTok{ mot16\_\_12,}
         \AttributeTok{mot16\_\_15 =} \DecValTok{8} \SpecialCharTok{{-}}\NormalTok{ mot16\_\_15,}
         \AttributeTok{mot16\_\_16 =} \DecValTok{8} \SpecialCharTok{{-}}\NormalTok{ mot16\_\_16)}
\end{Highlighting}
\end{Shaded}

\begin{quote}
Le code paraît fastidieux et potentiellement source d'erreurs. De plus,
ce recodage s'accompagne d'une feuille annexe sur laquelle on a noté
avec prudence les items en cause et la formule pour inverser les
scores\ldots{} On ne sait mieux faire\ldots{}
\end{quote}

Ce procédé d'écrasement facilite l'analyse de la cohérence interne ou le
calcul des scores par la suite (cf.~chapitre \emph{description}).

\hypertarget{calcul-du-score-de-chaque-questionnaire-par-observation}{%
\subsection{calcul du score de chaque questionnaire par
observation}\label{calcul-du-score-de-chaque-questionnaire-par-observation}}

La variable qui contient le score global de chaque questionnaire par
participant·e porte l'extension *\_sco* :

\begin{quote}
Exemple : \texttt{mot13\_sco} est la variable de score total de
\texttt{mot13\_1} à \texttt{mot13\_13} avec les variables qui ont été
inversées (sans conservation dans le df de l'item non-inversé).
\end{quote}

\begin{Shaded}
\begin{Highlighting}[]
\CommentTok{\#Recondage des variables au score inversé}

\CommentTok{\#Création des moyennes de chaque questionnaire pour chaque observation}

\NormalTok{d\_long }\OtherTok{\textless{}{-}}\NormalTok{ d\_long }\SpecialCharTok{\%\textgreater{}\%} 
  \FunctionTok{mutate}\NormalTok{(}\AttributeTok{hbs\_sco =} \FunctionTok{rowMeans}\NormalTok{(}\FunctionTok{select}\NormalTok{(.,}\FunctionTok{starts\_with}\NormalTok{(}\StringTok{"hbs"}\NormalTok{)),}\AttributeTok{na.rm =}\NormalTok{T),}
         \AttributeTok{pec\_sco =} \FunctionTok{rowMeans}\NormalTok{(}\FunctionTok{select}\NormalTok{(.,}\FunctionTok{starts\_with}\NormalTok{(}\StringTok{"pec"}\NormalTok{)),}\AttributeTok{na.rm =}\NormalTok{T),}
         \AttributeTok{be\_sco  =} \FunctionTok{rowMeans}\NormalTok{(}\FunctionTok{select}\NormalTok{(.,}\FunctionTok{starts\_with}\NormalTok{(}\StringTok{"be"}\NormalTok{)) ,}\AttributeTok{na.rm =}\NormalTok{T),}
         \AttributeTok{est\_sco =} \FunctionTok{rowMeans}\NormalTok{(}\FunctionTok{select}\NormalTok{(.,}\FunctionTok{starts\_with}\NormalTok{(}\StringTok{"est"}\NormalTok{)),}\AttributeTok{na.rm =}\NormalTok{T),}
         \AttributeTok{cli\_sco =} \FunctionTok{rowMeans}\NormalTok{(}\FunctionTok{select}\NormalTok{(.,}\FunctionTok{starts\_with}\NormalTok{(}\StringTok{"cli"}\NormalTok{)),}\AttributeTok{na.rm =}\NormalTok{T),}
         \AttributeTok{sou\_sco =} \FunctionTok{rowMeans}\NormalTok{(}\FunctionTok{select}\NormalTok{(.,}\FunctionTok{starts\_with}\NormalTok{(}\StringTok{"sou"}\NormalTok{)),}\AttributeTok{na.rm =}\NormalTok{T),}
         \AttributeTok{mot\_sco =} \FunctionTok{rowMeans}\NormalTok{(}\FunctionTok{select}\NormalTok{(.,}\FunctionTok{starts\_with}\NormalTok{(}\StringTok{"hbs"}\NormalTok{)),}\AttributeTok{na.rm =}\NormalTok{T))}
\end{Highlighting}
\end{Shaded}

\begin{quote}
De nouveau, on ne crée pas d'objet intermédiaire sachant que R sait très
bien nous permettre de vérifier le processus, étape par étape.
\end{quote}

Ce code paraît très efficace. Il se base sur notre nomenclature sans
risque d'erreur, indépendamment du nombre de variable ou de la position
des colonnes.

\begin{quote}
On apprécie, même si on ne comprend pas encore à satisfaction la
fonction ´select´et son fonctionnement.
\end{quote}

\hypertarget{filtrage-des-observations}{%
\subsection{filtrage des observations}\label{filtrage-des-observations}}

Ici, on a essayé de développer une technique pour ne garder que les
\emph{id} qui se retrouvent strictement au temps 1 et au temps 2.

Les enjeux sont cruciaux :

\begin{itemize}
\tightlist
\item
  on doit s'assurer qu'un même \emph{id} ne se retrouve pas plusieurs
  fois dans le même temps (un·e participant·e peut avoir rempli 2 fois
  le questionnaire du temps, p.ex.)
\item
  on doit aussi s'assurer que nous ferons nos comparaisons temps 1 /
  temps 2 sur des données complètes
\end{itemize}

C'est notre présomption de base. Il est dès lors important de filtrer
les observations de manière stricte.

Pour cela, nous avons :

\begin{itemize}
\tightlist
\item
  créé un df de comparaison pour pouvoir dire à R quoi sélectionner. Ce
  df va, en deux étapes, ne garder que (1) les \emph{id} uniques dans
  chaque temps puis (2) constituant une paire (t1, t2).
\end{itemize}

\begin{Shaded}
\begin{Highlighting}[]
\NormalTok{d\_comp }\OtherTok{\textless{}{-}}\NormalTok{ d\_long }\SpecialCharTok{\%\textgreater{}\%} 
  \FunctionTok{drop\_na}\NormalTok{(id) }\SpecialCharTok{\%\textgreater{}\%} \CommentTok{\#par sécurité, on supprime les lignes dont l\textquotesingle{}id est vide}
  \FunctionTok{arrange}\NormalTok{(id) }\SpecialCharTok{\%\textgreater{}\%} \CommentTok{\#visuel uniquement}
  \FunctionTok{group\_by}\NormalTok{(id, temps) }\SpecialCharTok{\%\textgreater{}\%} 
  \FunctionTok{count}\NormalTok{(id) }\SpecialCharTok{\%\textgreater{}\%} 
  \FunctionTok{filter}\NormalTok{(n}\SpecialCharTok{==}\DecValTok{1}\NormalTok{) }\SpecialCharTok{\%\textgreater{}\%} \CommentTok{\#On s\textquotesingle{}est assuré que chaque id est unique dans chaque modalité de temps. On doit encore être sûrs qu\textquotesingle{}on a maintenant exactement une paire (t1,t2). Donc on continue le processus que R réalise dans cet ordre.}
  \FunctionTok{ungroup}\NormalTok{() }\SpecialCharTok{\%\textgreater{}\%} 
  \FunctionTok{group\_by}\NormalTok{(id) }\SpecialCharTok{\%\textgreater{}\%} 
  \FunctionTok{count}\NormalTok{(id) }\SpecialCharTok{\%\textgreater{}\%} 
  \FunctionTok{filter}\NormalTok{(n}\SpecialCharTok{==}\DecValTok{2}\NormalTok{) }\SpecialCharTok{\%\textgreater{}\%} \CommentTok{\#on ne garde que les paires de id qui se retrouvent dans t1 et t2. C\textquotesingle{}est notre grosse perte de données de ce traitement}
  \FunctionTok{ungroup}\NormalTok{()}
\end{Highlighting}
\end{Shaded}

\begin{quote}
On devrait vérifier la validité de cette méthode, sur le plan des bonnes
pratiques stat, et aussi sur le plan du codage dans R\ldots et aussi
s'assurer qu'elle fonctionne bien comme on le pense en la mettant à
l'épreuve.
\end{quote}

\begin{itemize}
\tightlist
\item
  créé un nouveau df qui ne contient plus que les paires souhaitées
\end{itemize}

\begin{Shaded}
\begin{Highlighting}[]
\CommentTok{\#Notre df de comparaison est prêt. On peut procéder à l\textquotesingle{}élagage de d\_long.}

\NormalTok{d\_long\_paired }\OtherTok{\textless{}{-}}\NormalTok{ d\_long }\SpecialCharTok{\%\textgreater{}\%} 
  \FunctionTok{filter}\NormalTok{(id }\SpecialCharTok{\%in\%}\NormalTok{ d\_comp}\SpecialCharTok{$}\NormalTok{id) }\SpecialCharTok{\%\textgreater{}\%} 
  \FunctionTok{mutate}\NormalTok{(}\AttributeTok{group=}\FunctionTok{ifelse}\NormalTok{(group}\SpecialCharTok{==}\StringTok{"con"}\NormalTok{, }\StringTok{"contrôle"}\NormalTok{,}\StringTok{"expérimental"}\NormalTok{))}
\end{Highlighting}
\end{Shaded}

\begin{quote}
On a été un peu sauvé avec le logique \texttt{\%in\%} mais sans bien
savoir pourquoi \texttt{==} ne convient pas. Dans l'example, on en a
profité pour modifier les facteurs de la variable \texttt{group}. Les
forums ne sont pas clairs sur la différence entre \texttt{ifelse}et
\texttt{if\_else}ou encore \texttt{case\_when}. On doit clarifier cela.
\end{quote}

\hypertarget{gestion-des-donnuxe9es-manquantes}{%
\subsection{gestion des données
manquantes}\label{gestion-des-donnuxe9es-manquantes}}

\begin{quote}
pas encore de bonnes pratiques.
\end{quote}

\hypertarget{gestion-des-donnuxe9es-extruxeames}{%
\subsection{gestion des données
extrêmes}\label{gestion-des-donnuxe9es-extruxeames}}

\begin{quote}
pas encore de bonnes pratiques.
\end{quote}

\bookmarksetup{startatroot}

\hypertarget{description-des-donnuxe9es}{%
\chapter{Description des données}\label{description-des-donnuxe9es}}

\hypertarget{obtenir-une-synthuxe8se-de-donnuxe9es}{%
\section{Obtenir une synthèse de
données}\label{obtenir-une-synthuxe8se-de-donnuxe9es}}

Voici deux exemples pour obtenir des petits résumés.

\begin{Shaded}
\begin{Highlighting}[]
\NormalTok{d\_long\_paired\_sum }\OtherTok{\textless{}{-}}\NormalTok{ d\_long\_paired }\SpecialCharTok{\%\textgreater{}\%} 
   \FunctionTok{mutate}\NormalTok{(}\AttributeTok{sex=}\FunctionTok{ifelse}\NormalTok{(sex}\SpecialCharTok{==}\StringTok{"1"}\NormalTok{, }\StringTok{"garçons"}\NormalTok{,}\StringTok{"filles"}\NormalTok{)) }\SpecialCharTok{\%\textgreater{}\%} 
   \FunctionTok{group\_by}\NormalTok{(temps, group, sex) }\SpecialCharTok{\%\textgreater{}\%} 
  \FunctionTok{summarise}\NormalTok{(}\AttributeTok{n=}\FunctionTok{n}\NormalTok{())}

\NormalTok{d\_long\_paired\_sum2 }\OtherTok{\textless{}{-}}\NormalTok{ d\_long\_paired }\SpecialCharTok{\%\textgreater{}\%} 
  \FunctionTok{group\_by}\NormalTok{(temps, group) }\SpecialCharTok{\%\textgreater{}\%} 
  \FunctionTok{summarise}\NormalTok{(}\AttributeTok{n=}\FunctionTok{n}\NormalTok{(),}
            \AttributeTok{mean\_hbs=}\FunctionTok{mean}\NormalTok{(hbs\_sco),}
            \AttributeTok{mean\_pec=}\FunctionTok{mean}\NormalTok{(pec\_sco),}
            \AttributeTok{mean\_be=}\FunctionTok{mean}\NormalTok{(be\_sco),}
            \AttributeTok{mean\_est=}\FunctionTok{mean}\NormalTok{(est\_sco),}
            \AttributeTok{mean\_cli=}\FunctionTok{mean}\NormalTok{(cli\_sco),}
            \AttributeTok{mean\_sou=}\FunctionTok{mean}\NormalTok{(sou\_sco),}
            \AttributeTok{mean\_mot=}\FunctionTok{mean}\NormalTok{(mot\_sco),}
            \AttributeTok{mean\_sho1=}\FunctionTok{mean}\NormalTok{(sho\_1),}
            \AttributeTok{mean\_sho2=}\FunctionTok{mean}\NormalTok{(sho\_2),}
            \AttributeTok{mean\_sho3=}\FunctionTok{mean}\NormalTok{(sho\_3),}
            \AttributeTok{mean\_sho4=}\FunctionTok{mean}\NormalTok{(sho\_4))}
\end{Highlighting}
\end{Shaded}

\hypertarget{principes}{%
\section{Principes}\label{principes}}

\hypertarget{procuxe9dures}{%
\section{Procédures}\label{procuxe9dures}}

\hypertarget{le-cas-des-boucles}{%
\section{Le cas des boucles}\label{le-cas-des-boucles}}

La fonction \texttt{for} semble très utile pour faire faire des boucles.
Quand je dois réaliser 10 plots de 10 VI, je peux juste préparer mon
plot et l'intégrer dans une boucle. La contrainte semble être liée à
\texttt{aes} et peut-être aussi au titre du plot dont on aimerait aussi
``automatiser'' la génération.

Infos à comprendre et tester ici :
\url{https://statistique-et-logiciel-r.com/comment-utiliser-ggplot-dans-une-boucle-ou-dans-une-fonction/}

intégrer et expliquer le cas d'école réussi avec les données IBE:

\begin{Shaded}
\begin{Highlighting}[]
\DocumentationTok{\#\#\#\#\#\#\#\#\#\#\#\#\#\#\#\#\#\#\#\#\#\#\#}
\CommentTok{\#visualisation en loop\#}
\DocumentationTok{\#\#\#\#\#\#\#\#\#\#\#\#\#\#\#\#\#\#\#\#\#\#\#}
\CommentTok{\# Liste des noms des variables que l\textquotesingle{}on veut (a à p uniquement).}
\NormalTok{var\_list }\OtherTok{=} \FunctionTok{names}\NormalTok{(d\_paired)[}\DecValTok{6}\SpecialCharTok{:}\DecValTok{21}\NormalTok{]}

\CommentTok{\# création de la liste pour accueillir les 16 (21{-}5) plots}
\NormalTok{plot\_list }\OtherTok{=} \FunctionTok{list}\NormalTok{()}

\ControlFlowTok{for}\NormalTok{ (i }\ControlFlowTok{in} \DecValTok{1}\SpecialCharTok{:}\DecValTok{16}\NormalTok{) \{}
\NormalTok{  p }\OtherTok{\textless{}{-}}\NormalTok{ d\_paired }\SpecialCharTok{\%\textgreater{}\%} 
    \FunctionTok{ggplot}\NormalTok{() }\SpecialCharTok{+}
    \FunctionTok{aes}\NormalTok{(}\AttributeTok{x =}\NormalTok{ dat) }\SpecialCharTok{+}
    \FunctionTok{aes\_string}\NormalTok{(}\AttributeTok{y =}\NormalTok{ var\_list[i]) }\SpecialCharTok{+}
    \FunctionTok{geom\_jitter}\NormalTok{(}\AttributeTok{size =} \DecValTok{5}\NormalTok{, }\AttributeTok{alpha =}\NormalTok{ .}\DecValTok{5}\NormalTok{, }\AttributeTok{width =} \FloatTok{0.3}\NormalTok{) }\SpecialCharTok{+}
    \FunctionTok{stat\_summary}\NormalTok{(}\AttributeTok{fun =}\NormalTok{ mean, }\AttributeTok{geom =} \StringTok{"point"}\NormalTok{, }\AttributeTok{size =} \DecValTok{3}\NormalTok{, }\AttributeTok{shape =} \DecValTok{4}\NormalTok{, }\AttributeTok{color =} \StringTok{"red"}\NormalTok{) }\SpecialCharTok{+}
    \FunctionTok{stat\_summary}\NormalTok{(}\AttributeTok{fun =}\NormalTok{ mean, }\AttributeTok{geom =} \StringTok{"line"}\NormalTok{, }\FunctionTok{aes}\NormalTok{(}\AttributeTok{group =} \DecValTok{1}\NormalTok{), }\AttributeTok{color =} \StringTok{"red"}\NormalTok{) }\SpecialCharTok{+}
    \FunctionTok{labs}\NormalTok{(}\AttributeTok{title =} \FunctionTok{paste}\NormalTok{(}\StringTok{"Mesure item"}\NormalTok{,i), }\AttributeTok{y =} \StringTok{"Score"}\NormalTok{) }\SpecialCharTok{+}
    \FunctionTok{theme}\NormalTok{(}\AttributeTok{plot.title =} \FunctionTok{element\_text}\NormalTok{(}\AttributeTok{hjust =} \FloatTok{0.5}\NormalTok{))  }
\NormalTok{  plot\_list[[i]] }\OtherTok{=}\NormalTok{ p}
\NormalTok{\}}
\FunctionTok{dev.off}\NormalTok{()}

\CommentTok{\# enregistrement des plots en png par fichier séparé avec un nom correspondant au nom de la variable et non de son numéro.}
\ControlFlowTok{for}\NormalTok{ (i }\ControlFlowTok{in} \DecValTok{1}\SpecialCharTok{:}\DecValTok{16}\NormalTok{) \{}
\NormalTok{  temp\_plot }\OtherTok{=}\NormalTok{ plot\_list[[i]]}
  \FunctionTok{ggsave}\NormalTok{(temp\_plot, }\AttributeTok{file=}\FunctionTok{paste0}\NormalTok{(}\StringTok{"plot\_"}\NormalTok{, var\_list[[i]],}\StringTok{".png"}\NormalTok{), }\AttributeTok{width =} \DecValTok{14}\NormalTok{, }\AttributeTok{height =} \DecValTok{10}\NormalTok{, }\AttributeTok{units =} \StringTok{"cm"}\NormalTok{)}
\NormalTok{\}}
\end{Highlighting}
\end{Shaded}

\hypertarget{tableaux}{%
\section{Tableaux}\label{tableaux}}

\hypertarget{plots}{%
\section{Plots}\label{plots}}

intégrer mes trouvailles de crips2019 et ce qu'un plot doit montrer.

Travail avec Zoe. Notes à intégrer :

Réaliser des boxplots avec barres d'erreurs

{[} {]} --\textgreater{} plutôt utiliser select des variables ou filter
des participants gather --\textgreater{} 1er argument nom et 2ème
argument valeur

set.sid dans Rmarkdown pour figer les aléatoires, p.ex. le jitter pour
pas que ça change à chaque lancement.

installer des kits de couleurs au besoin.

shape : les numéros correspondent à des formes de points.

geom\_line : 3 arguments minimum. avec group. --\textgreater{} pourquoi
``group'' et pourquoi ``1''. Alt + N = \textasciitilde{}

jitter --\textgreater{} position et positionjitterdodge pour comparer G
et F

Afficher toutes les catégries, même les vides :

\begin{Shaded}
\begin{Highlighting}[]
    \FunctionTok{scale\_x\_discrete}\NormalTok{(}\AttributeTok{drop =} \ConstantTok{FALSE}\NormalTok{) }\CommentTok{\# Forcer l\textquotesingle{}affichage des catégories vides}
\CommentTok{\# A voir si la variable doit être un facteur}
\NormalTok{d }\OtherTok{\textless{}{-}}\NormalTok{ d }\SpecialCharTok{\%\textgreater{}\%} 
  \FunctionTok{mutate}\NormalTok{(}
         \AttributeTok{q1 =} \FunctionTok{factor}\NormalTok{(q1, }\AttributeTok{levels =} \FunctionTok{c}\NormalTok{(}\StringTok{"1"}\NormalTok{,}\StringTok{"2"}\NormalTok{,}\StringTok{"3"}\NormalTok{,}\StringTok{"4"}\NormalTok{), }\AttributeTok{labels =} \FunctionTok{c}\NormalTok{(}\StringTok{"pas du tout"}\NormalTok{, }\StringTok{"plutôt non"}\NormalTok{, }\StringTok{"plutôt oui"}\NormalTok{, }\StringTok{"tout à fait"}\NormalTok{)),}
         \AttributeTok{q2 =} \FunctionTok{factor}\NormalTok{(q2, }\AttributeTok{levels =} \FunctionTok{c}\NormalTok{(}\StringTok{"1"}\NormalTok{,}\StringTok{"2"}\NormalTok{,}\StringTok{"3"}\NormalTok{,}\StringTok{"4"}\NormalTok{), }\AttributeTok{labels =} \FunctionTok{c}\NormalTok{(}\StringTok{"pas du tout"}\NormalTok{, }\StringTok{"plutôt non"}\NormalTok{, }\StringTok{"plutôt oui"}\NormalTok{, }\StringTok{"tout à fait"}\NormalTok{)),}
         \AttributeTok{q3 =} \FunctionTok{factor}\NormalTok{(q3, }\AttributeTok{levels =} \FunctionTok{c}\NormalTok{(}\StringTok{"1"}\NormalTok{,}\StringTok{"2"}\NormalTok{,}\StringTok{"3"}\NormalTok{,}\StringTok{"4"}\NormalTok{), }\AttributeTok{labels =} \FunctionTok{c}\NormalTok{(}\StringTok{"pas du tout"}\NormalTok{, }\StringTok{"plutôt non"}\NormalTok{, }\StringTok{"plutôt oui"}\NormalTok{, }\StringTok{"tout à fait"}\NormalTok{)),}
         \AttributeTok{q4 =} \FunctionTok{factor}\NormalTok{(q4, }\AttributeTok{levels =} \FunctionTok{c}\NormalTok{(}\StringTok{"1"}\NormalTok{,}\StringTok{"2"}\NormalTok{,}\StringTok{"3"}\NormalTok{,}\StringTok{"4"}\NormalTok{), }\AttributeTok{labels =} \FunctionTok{c}\NormalTok{(}\StringTok{"pas du tout"}\NormalTok{, }\StringTok{"plutôt non"}\NormalTok{, }\StringTok{"plutôt oui"}\NormalTok{, }\StringTok{"tout à fait"}\NormalTok{)),}
\NormalTok{  )}
\end{Highlighting}
\end{Shaded}

\bookmarksetup{startatroot}

\hypertarget{rapport}{%
\chapter{Rapport}\label{rapport}}

\hypertarget{principes-1}{%
\section{Principes}\label{principes-1}}

Avec ressources (fameux document PDF de CS durant le pre-doc) Ce qui est
attendu dans un document de type rapport. (à l'aide du rapport de ZL et
de celui en production pour LV)

Kable et KableExtra

bon affichage d'un tableau

insérer image externe

\hypertarget{procuxe9dures-cluxe9s}{%
\section{Procédures clés}\label{procuxe9dures-cluxe9s}}

\hypertarget{un-mot-sur-les-normes-apa}{%
\section{Un mot sur les normes APA}\label{un-mot-sur-les-normes-apa}}

\hypertarget{un-mot-sur-bookdown-et-github}{%
\section{Un mot sur Bookdown et
GitHub}\label{un-mot-sur-bookdown-et-github}}

cf.~chapitre 1.

\bookmarksetup{startatroot}

\hypertarget{etudes-de-cas}{%
\chapter{Etudes de cas}\label{etudes-de-cas}}

\hypertarget{data-masking-walrus-opuxe9rator-et-sting-interpolation}{%
\section{Data masking, Walrus opérator et sting
interpolation}\label{data-masking-walrus-opuxe9rator-et-sting-interpolation}}

Exemple de Mutate dans une fonction ou même au sein d'un loop. voir
sandbox.

\begin{Shaded}
\begin{Highlighting}[]
\CommentTok{\# avec tunnel. pas essayé encore.}
\NormalTok{fun }\OtherTok{\textless{}{-}} \ControlFlowTok{function}\NormalTok{(df, var) \{}
\NormalTok{  df }\OtherTok{\textless{}{-}}\NormalTok{ df }\SpecialCharTok{\%\textgreater{}\%} 
    \FunctionTok{mutate}\NormalTok{(}\StringTok{"\{var\}"} \SpecialCharTok{:}\ErrorTok{=}\NormalTok{ key }\SpecialCharTok{{-}}\NormalTok{ \{\{var\}\})}
\NormalTok{\}}
\CommentTok{\# avec pull(). OK.}

\NormalTok{fun }\OtherTok{\textless{}{-}} \ControlFlowTok{function}\NormalTok{(df, var) \{}
\NormalTok{  df }\OtherTok{\textless{}{-}}\NormalTok{ df }\SpecialCharTok{\%\textgreater{}\%} 
    \FunctionTok{mutate}\NormalTok{(}\StringTok{"\{var\}"} \SpecialCharTok{:}\ErrorTok{=}\NormalTok{ key }\SpecialCharTok{{-}} \FunctionTok{pull}\NormalTok{(.,var))}
\NormalTok{\}}
\end{Highlighting}
\end{Shaded}

\bookmarksetup{startatroot}

\hypertarget{ruxe9fuxe9rences}{%
\chapter*{Références}\label{ruxe9fuxe9rences}}
\addcontentsline{toc}{chapter}{Références}

\markboth{Références}{Références}

\hypertarget{refs}{}
\begin{CSLReferences}{0}{0}
\end{CSLReferences}



\end{document}
